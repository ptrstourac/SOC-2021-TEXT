\chapter{Integrace do výuky a využití}
Již v~průběhu psaní této práce jsou materiály z~výukové sady využívány vyučujícími a studenty naší školy.
Kromě nich je využívají i někteří vysokoškoláci a nestudující lidé, kteří se chtějí naučit modelovat.

\section{Sledovanost v~číslech}
\B{\textcolor{mygreen}{PŠ: Tuto sekci bych rád vztáhnul ke konkrétnímu datu a zmínil konkrétní čísla. Rád bych počkal ještě nějakou chvíli, mezitím by se sledovanost měla přehoupnout přes ty dva tisíce. TZN napíšu nejpozději ve středu večer... }\normalsize}

\section{Distanční výuka}
Výuková videa našla při distanční výuce podstatné využití.
Videokonferenční programy nezaručují dostatečně vysokou kvalitu přenosu obrazu a zvuku, kvůli čemuž by musel vyučující ukázku postupu při výpadku na straně studentů několikrát opakovat.
Učitelé na naší škole proto v~některých případech vlastní ukázky zcela nahrazovali odkazováním na mnou vytvořená výuková videa a materiály.

\section{Využití materiálů v~prezenční výuce}
K~praktickému využití v~prezenční výuce bohužel z~důvodu celosvětové pandemie zatím nedošlo. 
I~přesto je jejich aplikace do výuky strojírenské konstrukce domluvená s~vyučujícími na naší škole.
Většina z~nich je plánuje s~obnovením prezenční výuky začít více využívat a v~příštím roce tuto aplikaci ještě více rozvinout.

\section{Další rozvoj}
\B{\textcolor{mygreen}{PŠ: Možná odsunout do závěru?}}\newline
Témat, která by si zasloužila začlenění do mé výukové sady existuje obrovské množství.
Vzhledem k~tomu, že mne tato tvorba opravdu baví, plánuji v~ní pokračovat i nadále.
Během následujících několika měsíců plánuji sadu rozšířit o~materiály zabývající se například tvorbou plechových dílů, svařovaných konstrukcí, nebo některých dalších prvků technické dokumentace.
