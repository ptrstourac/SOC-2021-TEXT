\chapter{Integrace do výuky a využití}
Již v~průběhu psaní této práce jsou materiály z~výukové sady využívány vyučujícími a studenty naší školy.
Kromě nich je využívají i někteří vysokoškoláci a nestudující lidé, kteří se chtějí naučit modelovat.

\section{Sledovanost v~číslech}
\B{\textcolor{mygreen}{PŠ: Tuto sekci bych rád vztáhnul ke konkrétnímu datu a zmínil konkrétní čísla. Rád bych počkal ještě nějakou chvíli, mezitím by se sledovanost měla přehoupnout přes ty dva tisíce. TZN napíšu nejpozději ve středu večer... }\normalsize}

\section{Distanční výuka}
Výuková videa našla při distanční výuce podstatné využití.
Videokonferenční programy nezaručují dostatečně vysokou kvalitu přenosu obrazu ani zvuku, kvůli čemuž by musel vyučující ukázku postupu při výpadku na straně studentů několikrát opakovat.
Učitelé na naší škole proto v~některých případech vlastní ukázky zcela nahrazovali odkazováním na mnou vytvořená výuková videa a materiály, jelikož jsou schopny tyto ukázky plně zastoupit.
\B{\textcolor{red}{PŠ: DOPSAT PÁR VĚT.}}

\section{Využití materiálů v~prezenční výuce}
K~praktickému využití v~prezenční výuce bohužel z~důvodu celosvětové pandemie zatím nedošlo. 
I~přesto je jejich aplikace do výuky strojírenské konstrukce plánovaná přinejmenším na naší škole.
Většina vyučujících je začne s~obnovením prezenční výuky využívat a v příštím roce bych praktickou aplikaci svých materiálů dále rozšířil.

Vzhledem k tomu, že jsou všechny materiály volně dostupné, jejich využití je proto možné i na jiných školách.
S tím se pojí i velmi snadné začlenění do výuky.
Vyučující nemusí nic složitě hledat -- na webových stránkách snadno a rychle najde vytisknutelné materiály i výuková videa.
Textové materiály samozřejmě není nutné používat jen v papírové podobě. Studentům je může snadno rozeslat v PDF, popřípadě může jen odkázat na webové stránky \href{https://www.p3dportal.cz}{www.p3dportal.cz}.

Způsobů využití mojí výukové sady v prezenční výuce existuje celá řada.
Vyučující upřednostňující textovou formu mohou studentům vytisknout a rozdat materiály, nebo jim je snadno rozeslat.
Pokud je v učebně k dispozici projektor, mohou promítat výuková videa, případně nechat studenty, aby je sledovali samostatně na svých počítačích.
Další možností je kombinované využití výukových videí a tištěných materiálů, kdy vyučující pomocí projektoru nechá studenty zhlédnout video, následně jim rozdá vytisknuté materiály a nechá je vypracovat úkoly a otázky, které jsou v nich obsaženy.