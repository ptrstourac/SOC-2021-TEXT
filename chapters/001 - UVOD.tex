\chapter*{Úvod}
\addcontentsline{toc}{chapter}{Úvod}
Počítačově asistovaný návrh je dnes nedílnou součástí strojírenské praxe.
Není proto divu, že se práce s CAD\footnote{Computer assisted design - počítačově asistovaný design} programy běžně vyučuje na odborných školách s technickým zaměřením.
Časová dotace těchto předmětů se zpravidla pohybuje okolo 2 až 4 hodin týdně, a liší se mezi jednotlivými školami, i mezi obory.
Přestože se jedná o jeden ze stěžejních předmětů, existuje pro něj velmi málo výukových materiálů.
Příprava materiálů proto velmi záleží přímo na samotných vyučujících.

Pro výuku SolidWorks, který je jedním z nejčastěji vyučovaných CADů aktuálně existuje pouze jedna učebnice.
Na druhou stranu videonávodů existuje mnohem více, zpravidla však nejsou primárně určeny pro použití ve výuce.

3D modelování mne odjakživa bavilo, při nástupu na střední školu pro mne tedy nešlo o nic nového.
Totéž se ovšem nedalo říci o spoustě mých spolužáků, kteří s ním měli velké problémy.
Často jsem se proto dostával do situace, kdy se blížil termín odevzdání nějakého projektu a já jsem byl doslova \enquote{zasypáván} dotazy spolužáků na to, jak vymodelovat nějaký prvek, popřípadě součást.
Pokaždé, když se nějaký konkrétní dotaz opakoval neustále dokola jsem přemýšlel, zda by neexistoval efektivnější způsob, jak spolužákům pomoci.
Začal jsem tedy odpovědi společně s ukázkami v SolidWorks natáčet.
V této počáteční fázi jsem však netušil, jak se celý projekt rozroste.

Postupně jsem začal uvažovat nad tím, zda by tato videa bylo možné využít i při výuce.
Konzultoval jsem tedy tento nápad s Ing. Zavadilem, který na naší škole učí předmět Konstrukční cvičení.
Shodli jsme se, že vytvoření videonávodů by ulehčilo práci nejen studentům, ale i vyučujícím.
V průběhu tvorby těchto videí jsem projekt postupně rozšiřoval a přidával další prvky, jako jsou tištěné materiály s otázkami a úkoly, nebo webový portál, aby bylo možné najít dílčí části na jednom místě.

Důležité je zmínit, že pro tvorbu určité součásti, nebo prvku může existovat více než jedno konkrétní řešení a není tedy jasně dáno, které z nich je správné.
Rozdíl mezi nimi je především v časové náročnosti a efektivitě.
Celá tato sada má tedy za cíl ukázat studentům optimální způsob řešení daného problému a následně jeho pochopení ověřit pomocí doplňujících úkolů a otázek.
V případě, že student kvůli přibývajícímu počtu postupů některý z nich zapomene, může se snadno vrátit a zpětně shlédnout videonávod, který problematiku popisuje, nebo nahlédnout do souvisejících doplňkových materiálů.

\newpage
