\chapter*{Úvod}
\addcontentsline{toc}{chapter}{Úvod}
Počítačově asistovaný návrh je dnes nedílnou součástí strojírenské praxe.
Není proto divu, že se práce s~CAD\footnote{Computer aided design - počítačově asistovaný design} programy běžně vyučuje na odborných školách s~technickým zaměřením.
Časová dotace těchto předmětů se zpravidla pohybuje okolo 2 až 4 hodin týdně, a liší se mezi jednotlivými školami, i mezi obory.
Přestože se jedná o~jeden ze stěžejních předmětů, existuje pro něj velmi málo výukových materiálů.
Příprava materiálů proto velmi záleží přímo na samotných vyučujících.

Pro výuku SolidWorks\cite{SOLIDWORKS}, který je jedním z~nejčastěji používaných CADů aktuálně existuje pouze jedna učebnice.
Na druhou stranu videonávodů existuje mnohem více, zpravidla však nejsou primárně určeny pro použití ve výuce.

3D modelování mne odjakživa bavilo, při nástupu na střední školu pro mne tedy nešlo o~nic nového.
Totéž se ovšem nedalo říci o~spoustě mých spolužáků, kteří s~ním měli velké problémy.
Často jsem se proto dostával do situace, kdy se blížil termín odevzdání nějakého projektu a já jsem byl doslova \enquote{zasypáván} dotazy spolužáků na to, jak vymodelovat nějaký prvek, popřípadě součást.
Pokaždé, když se nějaký konkrétní dotaz opakoval, jsem přemýšlel, zda by neexistoval efektivnější způsob, jak spolužákům pomoci.
Začal jsem proto odpovědi společně s~ukázkami v~SolidWorks natáčet.
V~této počáteční fázi jsem však netušil, jak se celý projekt s přibývajícím časem rozroste.

Postupně jsem začal uvažovat nad tím, zda by tato videa bylo možné využít i při výuce.
Konzultoval jsem tento nápad s~Ing. Zavadilem, který na naší škole učí předmět konstrukční cvičení.
Shodli jsme se, že vytvoření výukových videí by ulehčilo práci nejen studentům, ale i vyučujícím.

V~průběhu tvorby těchto videí jsem projekt postupně rozšiřoval a přidával další, neméně důležité prvky.
Prvním z~nich byly tisknutelné textové materiály.
Uvědomil jsem si totiž, že ne každému studentovi může audiovizuální forma předávání informací vyhovovat.
K~výukovým videím jsem z toho důvodu vytvořil doplňkové vytisknutelné materiály, které předávají obsah videí v~textové podobě.
Textové materiály jsem postupně rozšířil i o~otázky a úkoly, pomocí kterých si může student dané téma zopakovat, nebo je může vyučující použít pro ověření zvládnutí získaných znalostí studenta.
Vznikla tak komplexní sada materiálů pro podporu výuky strojírenské konstrukce v~SolidWorks.
Proto, aby byla celá sada snadno a rychle dostupná jsem následně přidal i webový portál P3D, na kterém jsou všechny její dílčí části volně k~dispozici.

Důležité je zmínit, že pro tvorbu určité součásti, nebo prvku\footnote{Každý díl, popřípadě součást se skládá z určitých prvků, které definují její tvar} může existovat více než jedno konkrétní řešení a není jasně dáno, které z~nich je správné.
Rozdíl mezi řešeními spočívá především v jejich časové náročnosti a efektivitě.
Celá výuková sada má tedy za cíl ukázat studentům optimální způsob řešení daného problému, a následně jeho pochopení ověřit pomocí doplňujících úkolů a otázek.
V~případě, že student kvůli přibývajícímu počtu postupů některý z~nich zapomene, může se snadno vrátit a zpětně shlédnout video, které problematiku popisuje, nebo nahlédnout do souvisejících doplňkových materiálů.

\newpage
