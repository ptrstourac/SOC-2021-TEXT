\chapter*{Úvod}
\addcontentsline{toc}{chapter}{Úvod}
Počítačově asistovaný návrh je dnes nedílnou součástí strojírenské praxe.
Není proto divu, že se práce s CAD\footnote{Computer assisted design - počítačově asistovaný design} programy běžně vyučuje na odborných školách s technickým zaměřením.
Časová dotace těchto předmětů se zpravidla pohybuje okolo 2 až 4 hodin týdně, přičemž se liší jak mezi jednotlivými školami, tak i mezi obory.
Přesto, že se jedná o jeden ze stěžejních předmětů, existuje pro něj velmi málo výukových materiálů.
Příprava výuky je tak čistě na samotných vyučujících.

Pro výuku SolidWorks, který je jedním z nejčastěji vyučovaných CADů aktuálně existuje pouze jedna učebnice.
Videonávodů existuje sice mnohem více, zpravidla ale nejsou vhodné pro výuku na školách.

3D modelování mne odjakživa bavilo, při nástupu na střední školu pro mne tedy nešlo o nic nového.
Totéž se ovšem nedalo říci o spoustě mých spolužáků, kteří s ním měli velké problémy.
Nezřídka jsem se proto dostával do situace, kdy se blížil termín odevzdání nějakého projektu a já byl \enquote{zasypáván} dotazy spolužáků na to, jak vymodelovat nějaký prvek, popřípadě součást.
Pokaždé, když se nějaký konkrétní dotaz opakoval neustále dokola jsem přemýšlel, zda by nebyl jednodušší způsob, jak spolužákům pomoci.
Začal jsem tedy odpovědi společně s ukázkami v SolidWorks natáčet.
Tehdy šlo pouze o určitý způsob, jak neopakovat jednu odpověď několikrát za den.

Na konci třetího ročníku jsme dostali za úkol vybrat si téma naší maturitní práce.


\vfill
\noindent \B{OSNOVA:}
\begin{itemize}[topsep=0pt]
    \setlength\itemsep{0.05em}
    \item co jsou to CADy a proč se učí
    \item proč dělám to, co dělám
    \item co mají studenti aktuálně k dispozici (jen povrchně)
\end{itemize}

\newpage
