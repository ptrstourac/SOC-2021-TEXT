\chapter{Názorně -- demonstrační pomůcky}


\section{Trendy ve vzdělávání}

\section{Předvádění a pozorování}

\section{Instruktáž}
Ve školním prostředí je instruktáž jednou z často používaných názorně -- demonstračních výukových metod.
Pomocí vizuálních, zvukových, popřípadě audiovizuálních podnětů umožňuje studentům si osvojovat nové dovednosti a v kombinaci s metodami praktickými je uplatňovat do praxe.
Skládá se zpravidla z ukázky doplněné komentářem -- takzvanými instrukcemi.

\noindent V závislosti na používaných podnětech rozlišujeme několik typů instruktáže:
\begin{itemize}[topsep=0pt]
    \setlength\itemsep{0em}
    \item \B{Slovní instruktáž} využívá mluvených, popřípadě psaných instrukcí k popsání vysvětlované činnosti.
    \item \B{Audiovizuální instruktáž} kombinuje slovní instruktáž s praktickou ukázkou, nebo vizuálními podklady (obrázky, video).
    \item \B{Písemná instruktáž} je spojením slovní instruktáže a psané formy doplněné o ilustrace.
\end{itemize}

\subsection{Instruktáž v jednotlivých částech výukové sady}
Jak výuková videa, tak i vytisknutelné materiály tvoří určitou formu instruktáže.
V případě výukových videí se jedná o instruktáž audiovizuální, spojující mluvený komentář (slovní instrukce) s názornou ukázkou.
Tyto dvě části se navzájem doplňují.
Zatímco názorná ukázka daný postup předvádí, mluvený komentář přidává doplňující informace a usměrňuje pozornost studenta.

Tisknutelné materiály jsou poté formou písemné instruktáže.
V textové podobě jsou zde přítomny slovní pokyny doplněné o ilustrace částí postupu, které jsou slovně špatně popsatelné (například vzhled určitého dialogového okna).

\section{Práce s obsahem}

\newpage