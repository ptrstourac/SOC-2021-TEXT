\chapter{Názorně-demonstrační pomůcky}
V~této kapitole se zaměřuji na trendy ve výuce technických předmětů a metody při ní využívané.

\section{Trendy ve výuce technických předmětů}
Výuka technických předmětů vyžaduje velmi specifický přístup.
Aby student látku správně pochopil, je kromě teoretických znalostí nezbytná i ukázka jejich aplikace.
Nejběžnějším způsobem této ukázky jsou modely dílů, nebo mechanismů speciálně upravených tak, aby si student vytvořil komplexní představu o~jejich funkci a účelu.
Tyto ukázkové modely však mají své nevýhody. 
Jedna z~největších nevýhod bývá patrná převážně při předvádění větších mechanismů.
Často totiž ve výuce není prostor ani dostatek času na detailní rozebrání modelu, a studenti tak často vidí jen zevnějšek, nebo některé vnitřní části skrze speciálně prořezané otvory.
Tím může být vytvoření ucelené představy o daném mechanismu pro studenta značně ztíženo.

Stále častěji se proto využívá počítačových vizualizací nebo animací, které umožňují celý model pomocí několika kliknutí snadno rozebrat nebo některé části skrýt. 
Díky tomu mohou studenti vidět funkci určitých mechanismů i zevnitř, což je ve srovnání s~modely nespornou výhodou.

Běžná bývá také praktická výuka vyučovaná v~dílnách, popřípadě laboratořích, kde dochází k~propojení teorie s~praxí jako například u~technologie třískového obrábění.
Studenti se nejdříve v~hodinách strojírenské technologie dozví, jak třískové obrábění probíhá, jeho princip a metody.
Následně si studenti mohou v~dílnách toto obrábění sami prakticky vyzkoušet na soustruhu nebo na frézce.

\section{Předvádění a pozorování}
V~souvislosti s~již zmíněnými ukázkovými modely se často aplikuje právě metoda předvádění a pozorování.
Jedná-li se o~fyzický model, studenti si jej mohou prohlížet, přičemž vyučující k danému modelu provede výklad.
U~elektronických modelů nebo animací je situace podobná.
Rozdíl je v~tom, že model není fyzicky přítomen v~učebně, ale animace, popřípadě vizualizace, je promítána na plátno.

Tato metoda je vhodná zejména pro výuku technických předmětů, kde potřebují studenti znát princip, funkci a účel určité součásti nebo mechanismu. 
Pro výuku konstrukce v~CAD programech je vhodná jen okrajově -- studenti potřebují vědět, jak mechanismus funguje, aby s~ním v~SolidWorks mohli poté správně individuálně pracovat.
Obvykle se však jedná pouze o~vizuální ukázku funkce, samotný postup tvorby tohoto mechanismu studentům popíše až instruktáž.

\section{Instruktáž\superscript{\cite{SVEC}}}
Ve školním prostředí je instruktáž jednou z~často používaných názorně-demonstračních výukových metod.
Pomocí vizuálních, zvukových, popřípadě audiovizuálních podnětů umožňuje studentům si osvojit nové znalosti a v~kombinaci s~metodami praktickými je uplatnit v~praxi.
Skládá se zpravidla z~ukázky doplněné komentářem -- takzvanými instrukcemi.

\noindent V~závislosti na používaných podnětech je možné rozlišit několik typů instruktáže:
\begin{itemize}[topsep=0pt]
    \setlength\itemsep{0em}
    \item \B{Slovní instruktáž} využívá verbálních instrukcí k~popsání vysvětlované činnosti.
    \item \B{Audiovizuální instruktáž} kombinuje slovní instruktáž s~praktickou ukázkou nebo vizuálními podklady (obrázky, video).
    \item \B{Písemná instruktáž} je spojením slovní instruktáže a psané formy doplněné o~ilustrace.
\end{itemize}

Instruktáž je nejčastěji využívána právě při výuce konstrukce v~SolidWorks, při které vyučující nejprve daný postup předvede a popíše, a poté si jej studenti sami vyzkoušejí.

\subsection{Instruktáž v~jednotlivých částech výukové sady}
Výuková videa i tisknutelné materiály jsou jistou formou instruktáže.
V~případě výukových videí se jedná o~instruktáž audiovizuální, spojující mluvený komentář (slovní instrukce) s~názornou ukázkou.
Tyto dvě části se navzájem doplňují a efektivně předávají studentovi ucelenou informaci.
Zatímco názorná ukázka daný postup předvádí, mluvený komentář přidává rozšiřující informace a usměrňuje pozornost studenta.

Tisknutelné materiály jsou formou písemné instruktáže.
V~textové podobě jsou zde přítomny slovní pokyny doplněné o~ilustrace částí postupu, které jsou pro tento typ úloh stěžejní.

\newpage