\chapter{Názorně -- demonstrační pomůcky}
V~této kapitole se zaměřuji na trendy ve výuce technických předmětů a metody při ní využívané.

\section{Trendy ve výuce technických předmětů}
Výuka technických předmětů vyžaduje velmi specifický přístup.
Aby student látku správně pochopil je kromě teoretických znalostí nezbytná i ukázka jejich praktické aplikace.
Nejběžnějším způsobem této ukázky jsou modely, díly, nebo mechanismy speciálně upravené tak, aby si student udělal komplexní představu o~jejich funkci a účelu.
Tyto ukázkové modely však mají své nevýhody. 
Patrně největší z~nich bývá patrná převážně při předvádění větších mechanismů.
Často totiž ve výuce není prostor a čas na detailní rozebrání modelu a studenti tak často vidí jen zevnějšek, nebo některé vnitřní části skrz speciálně prořezané otvory.

Stále častěji se proto využívá počítačových vizualizací, nebo animací, které umožňují celý model pomocí několika kliknutí snadno rozebrat, nebo některé části skrýt. 
Díky tomu mohou studenti vidět funkci některých mechanismů zevnitř, což by u~modelu nemuselo být možné.

Častá bývá také praktická výuka vyučovaná v~dílnách, popřípadě laboratořích, kde dochází k~propojení teorie s~praxí.
Pro příklad můžeme vzít třeba technologii obrábění.
Studenti se nejdříve v~hodinách strojírenské technologie dozví, jak třískové obrábění probíhá, jeho princip a metody.
Následně si studenti mohou v~dílnách toto obrábění vyzkoušet -- ať již na soustruhu, či na frézce. 

\section{Předvádění a pozorování}
V~souvislosti s~již zmíněnými ukázkovými modely se často aplikuje právě předvádění a pozorování.
Jedná-li se o~fyzický model, studenti si jej mohou prohlížet, přičemž vyučující o~daném modelu provede výklad.
U~elektronických modelů, nebo animací je to podobné, jen s~rozdílem že model není fyzicky v~učebně, ale je promítán třídě na plátno.

Tato metoda je vhodná primárně pro výuku technických předmětů, kde potřebují studenti znát princip, funkci a účel určité součásti, nebo mechanismu. 
Pro výuku konstrukce v~CAD programech je vhodná jen okrajově -- studenti potřebují vědět, jak mechanismus funguje, aby s~ním v~SolidWorks mohli správně pracovat.
Samotný postup tvorby tohoto mechanismu jim ale popíše až instruktáž.

\section{Instruktáž}
Ve školním prostředí je instruktáž jednou z~často používaných názorně -- demonstračních výukových metod.
Pomocí vizuálních, zvukových, popřípadě audiovizuálních podnětů umožňuje studentům si osvojovat nové dovednosti a v~kombinaci s~metodami praktickými je uplatňovat do praxe.
Skládá se zpravidla z~ukázky doplněné komentářem -- takzvanými instrukcemi.

\noindent V~závislosti na používaných podnětech lze rozlišit několik typů instruktáže:
\begin{itemize}[topsep=0pt]
    \setlength\itemsep{0em}
    \item \B{Slovní instruktáž} využívá verbálních instrukcí k~popsání vysvětlované činnosti.
    \item \B{Audiovizuální instruktáž} kombinuje slovní instruktáž s~praktickou ukázkou, nebo vizuálními podklady (obrázky, video).
    \item \B{Písemná instruktáž} je spojením slovní instruktáže a psané formy doplněné o~ilustrace.
\end{itemize}

Instruktáž je velice často používána právě při výuce konstrukce v~SolidWorks, kdy vyučující nejprve nějaký postup předvede a popíše a poté si jej studenti sami vyzkoušejí.

\subsection{Instruktáž v~jednotlivých částech výukové sady}
Jak výuková videa, tak i vytisknutelné materiály tvoří určitou formu instruktáže.
V~případě výukových videí se jedná o~instruktáž audiovizuální, spojující mluvený komentář (slovní instrukce) s~názornou ukázkou.
Tyto dvě části se navzájem doplňují.
Zatímco názorná ukázka daný postup předvádí, mluvený komentář přidává doplňující informace a usměrňuje pozornost studenta.

Tisknutelné materiály jsou poté formou písemné instruktáže.
V~textové podobě jsou zde přítomny slovní pokyny doplněné o~ilustrace částí postupu, které jsou slovně špatně popsatelné (například vzhled určitého dialogového okna).

\newpage