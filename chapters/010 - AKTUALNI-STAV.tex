\chapter{Dosavadní výuka strojírenské konstrukce}


\section{Způsob výuky konstrukce v SolidWorks}
    Výuka předmětů, jako je již zmíněné \enquote{konstrukční cvičení} probíhá na projektové bázi.
    Co si pod tímto pojmem představit?
    Vyučující seznámí studenty s projektem, na kterém budou v následujících týdnech, popř. měsících pracovat. 
    Tyto projekty jsou do roka většinou 3 až 4.
    Následně jim během několika vyučovacích hodin ukáže postup práce s prvky, které by mohli studenti při práci na projektu potřebovat.
    Poté již nechá studenty pracovat samostatně na svých projektech.
    V běžných vyučovacích hodinách proto tvoří samostatná práce zhruba 60 až 70 procent času.

    Samostatná práce v hodině má pro studenty výhodu v možnosti se vyučujícího zeptat, pokud něco neví, nebo s něčím mají problém.
    Většina studentů ale na projektech pracuje hlavně doma, kde možnost jednoduše se zeptat nemají.
    Vyučujícího sice kontaktovat mohou, ale málokdy toho využijí.

\section{Co mají studenti aktuálně k dispozici?}
    Na tuto otázku se dá snadno odpovědět jednou větou \enquote{No, moc toho opravdu není}.
    Pokud z výběru vyřadíme cizojazyčné publikace, existuje pouze jedna \It{Učebnice SolidWorks}\superscript{\cite{PAGAC}}, kterou tento seznam začíná a zároveň končí. 
    Videonávody na toto téma existují, zpravidla ale nejsou v souladu se školní praxí.
    Většinou jsou dlouhá (> 10 minut), přičemž student udrží pozornost zhruba 5 minut.
    Dalším problémem je jejich obsáhlost.
    Pokud student hledá postup tvorby konkrétního prvku, nechce kvůli tomu procházet desetiminutové video ve kterém tvoří hledaný obsah například jen 5\%. 
