\chapter{Dosavadní výuka strojírenské konstrukce}
Na začátek bych rád popsal, co to vlastně SolidWorks je, jak aktuálně probíhá výuka předmětů zaměřených na strojírenskou konstrukci v něm a materiály, které mají studenti k dispozici.

\section{Co to je SolidWorks a proč se používá?}
    CAD systém SolidWorks vyvíjený francouzskou společností Dassault Systémes je dnes jedním z nejpoužívanějších programů pro 3D modelování a tvorbu technické dokumentace ve strojírenství.
    Mimo již zmíněných funkcí je možné s jeho pomocí vytvářet i různé simulace (pevnostní i pohybové), spravovat data jednotlivých výrobků a jejich životní cyklus (PLM\footnote{Product lifecycle management -- správa životního cyklu výrobku} a PDM\footnote{Product Data Management -- správa dat výrobku}), připravovat výrobní data pro obrábění (CAM\footnote{Computer Aided Manufacturing -- počítačová podpora výroby (resp. obrábění)}) a mnoho dalšího.
    Vzhledem k jeho komplexnosti jsou pro praktickou práci s ním zapotřebí alespoň základní znalosti funkcí tohoto systému.

    Právě kvůli jeho rozšířenosti se práce s ním běžně vyučuje na mnoha středních průmyslových školách.
    Podstatnou výhodu pro studenty tvoří možnost získání tzv. SDK \footnote{Student Design Kit -- studentská licence bez jakýchkoliv doplň. modulů}, který jim umožňuje pracovat s 3D modely i doma.

    Za zmínku také stojí, že na některých školách mohou zdatnější studenti již v průběhu studia složit zkoušky pro získání certifikátů CSWA (Certified SolidWorks Associate), nebo CSWP (Certified SolidWorks Professional), díky kterým mohou získat lepší pozici při žádání o práci ve firmě, která SolidWorks používá.

\section{Způsob výuky práce v SolidWorks na naší škole}
    Výuka již zmíněného konstrukčního cvičení probíhá na projektové bázi.
    Co si pod tímto pojmem představit?
    Vyučující seznámí studenty s projektem, na kterém budou v následujících týdnech, popř. měsících pracovat. 
    Tyto projekty jsou do roka většinou 3 až 4.
    Následně je vyučující v průběhu několika hodin seznámí s postupem práce s prvky, které by mohli studenti při práci na projektu potřebovat.
    Poté poté studenti pracují samostatně na svých projektech.
    V běžných vyučovacích hodinách proto tvoří samostatná práce studentů většinu času.

    Jestliže studenti pracují v průběhu vyučovací hodiny samostatně, mohou interagovat s vyučujícím a s jeho pomocí dospět ke správnému řešení.
    Většina studentů však na projektech pracuje převážně doma, kde nemají možnost se jednoduše zeptat, pokud něčemu nerozumí, nebo nejsou schopni danou úlohu vyřešit.

\section{Co mají studenti aktuálně k dispozici?}
    Na tuto otázku se dá snadno odpovědět jednou větou \enquote{No, moc toho opravdu není}.
    Pokud z výběru vyřadíme cizojazyčné publikace, existuje pouze jedna \It{Učebnice SolidWorks}\superscript{\cite{PAGAC}}, kterou tento seznam začíná a zároveň končí. 
    Videonávody na toto téma existují, zpravidla však neodpovídají školní praxi
    Většinou jsou dlouhá (> 10 minut), a student nemusí udržet záměrnou pozornost po celou dobu.
    Dalším problémem je jejich obsáhlost.
    Pokud student hledá postup tvorby konkrétního prvku, není nutné, aby kvůli tomu zhlédnul celé desetiminutové video, ve kterém tvoří hledaný obsah například jen 5 procent. 
