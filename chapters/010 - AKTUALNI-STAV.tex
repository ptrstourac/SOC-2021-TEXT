\chapter{Dosavadní výuka strojírenské konstrukce}
\fxnote[inline=true]{\textcolor{mygreen}{Tato sekce je ještě částečně rozpracovaná, možná ji ještě přeformuluji celou}}

\section{Co to je SolidWorks a proč se používá?}

\section{Způsob výuky práce v SolidWorks na naší škole}
    Výuka již zmíněného konstrukčního cvičení probíhá na projektové bázi.
    Co si pod tímto pojmem představit?
    Vyučující seznámí studenty s projektem, na kterém budou v následujících týdnech, popř. měsících pracovat. 
    Tyto projekty jsou do roka většinou 3 až 4.
    Následně je vyučující v průběhu několika hodin seznámí s postupem práce s prvky, které by mohli studenti při práci na projektu potřebovat.
    Poté poté studenti pracují samostatně na svých projektech.
    V běžných vyučovacích hodinách proto tvoří samostatná práce studentů většinu času.

    Jestliže studenti pracují v průběhu vyučovací hodiny samostatně, mohou interagovat s vyučujícím a s jeho pomocí dospět ke správnému řešení.
    Většina studentů však na projektech pracuje převážně doma, kde nemají možnost se jednoduše zeptat, pokud něčemu nerozumí, nebo nejsou schopni danou úlohu vyřešit.

\section{Co mají studenti aktuálně k dispozici?}
    Na tuto otázku se dá snadno odpovědět jednou větou \enquote{No, moc toho opravdu není}.
    Pokud z výběru vyřadíme cizojazyčné publikace, existuje pouze jedna \It{Učebnice SolidWorks}\superscript{\cite{PAGAC}}, kterou tento seznam začíná a zároveň končí. 
    Videonávody na toto téma existují, zpravidla však neodpovídají školní praxi
    Většinou jsou dlouhá (> 10 minut), a student nemusí udržet záměrnou pozornost po celou dobu.
    Dalším problémem je jejich obsáhlost.
    Pokud student hledá postup tvorby konkrétního prvku, není nutné, aby kvůli tomu zhlédnul celé desetiminutové video, ve kterém tvoří hledaný obsah například jen 5 procent. 
