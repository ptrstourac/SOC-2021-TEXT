\chapter{Dosavadní výuka strojírenské konstrukce}
Na začátek bych rád popsal, co to SolidWorks je, jak aktuálně probíhá výuka předmětů zaměřených na strojírenskou konstrukci v~něm, a materiály, které mají studenti a vyučující k~dispozici.

\section{Co to je SolidWorks a proč se používá?}
    CAD systém SolidWorks vyvíjený francouzskou společností Dassault Systémes je dnes jedním z~nejpoužívanějších programů pro 3D modelování a tvorbu technické dokumentace ve strojírenství.
    Mimo již zmíněné funkce je možné s~jeho pomocí vytvářet i různé simulace (pevnostní i pohybové), spravovat data jednotlivých výrobků i jejich životní cyklus (PLM\footnote{Product lifecycle management -- správa životního cyklu výrobku} a PDM\footnote{Product Data Management -- správa dat výrobku}), připravovat výrobní data pro obrábění (CAM\footnote{Computer Aided Manufacturing -- počítačová podpora výroby (resp. obrábění)}) a mnoho dalšího.
    Vzhledem k~jeho komplexitě jsou pro praktickou práci s~ním zapotřebí alespoň základní znalosti funkcí tohoto systému.

    Právě kvůli rozšířenosti SolidWorks ve strojírenství se práce s~ním běžně vyučuje na mnoha středních průmyslových školách.
    Podstatnou výhodu pro studenty tvoří možnost získání tzv. SDK\footnote{studentská licence bez jakýchkoliv doplň. modulů} (Student design kit), který jim umožňuje pracovat s~3D modely i doma.

    Za zmínku také stojí, že na některých školách mohou zdatnější studenti již v~průběhu studia složit zkoušky pro získání certifikátu CSWA (Certified SolidWorks Associate), nebo CSWP (Certified SolidWorks Professional), díky kterým mohou později získat lepší výchozí pozici při žádosti o~práci ve firmě, která SolidWorks používá.

\section{Způsob výuky práce v~SolidWorks na naší škole}
    Výuka již zmíněného konstrukčního cvičení probíhá na projektové bázi.
    Co si pod tímto pojmem představit?
    Vyučující seznámí studenty s~projektem, na kterém budou v~následujících týdnech popř. měsících pracovat. 
    Tyto projekty jsou v~průběhu školního roku zpravidla 3 až 4.
    Následně jsou studenti seznámeni v~průběhu několika vyučovacích hodin s~postupem práce s~prvky, které by mohli při práci na projektu potřebovat.
    Poté pracují samostatně na svých projektech.
    Ve vyučovacích hodinách proto tvoří samostatná práce studentů většinu času.

    Jestliže studenti pracují v~průběhu vyučovací hodiny samostatně, mohou interagovat s~vyučujícím a s~jeho pomocí dospět ke správnému řešení.
    Mnozí studenti však na projektech pracují převážně doma, kde nemají možnost se jednoduše zeptat, pokud něčemu nerozumí, nebo nejsou schopni danou úlohu vyřešit.

\section{Výukové materiály související se strojírenskou konstrukcí}
    V dnešní době existuje pouze jedna učebnice, která se zabývá problematikou SolidWorks v češtině.
    Videonávody na toto téma existují, zpravidla však neodpovídají školní praxi a často ani nejsou pro výuku vhodná.
    Většinou jsou dlouhá (> 10 minut), a student nemusí udržet záměrnou pozornost po celou dobu.
    Dalším problematickým aspektem je jejich obsáhlost.
    Pokud student hledá postup tvorby konkrétního prvku, není nutné, aby kvůli tomu zhlédnul celé dvacetiminutové video, ve kterém tvoří hledaný obsah například jen 5 procent. 
    Jedná se o videonávody, které jsou primárně určeny k jinému účelu, než je využití ve výuce. 
