\chapter{Cíle práce}

\begin{enumerate}[topsep=0pt]
    \setlength\itemsep{0em}
    \item Usnadnění výuky strojírenské konstrukce v~programu SolidWorks
    \item Vytvoření komplexní výukové sady, jejíž cílem je studentům stručně, jasně a přehledně demonstrovat postupy tvorby různých prvků v SolidWorks. Tato sada se skládá z:
    \begin{itemize}[topsep=0pt]
        \setlength\itemsep{0em}
        \item Výukových videí fungujících jako audiovizuální instruktáž práce s různými prvky a funkcemi v konstrukčním programu SolidWorks. Již při sledování těchto videí si může student demonstrovaný postup samostatně zkoušet, nebo procvičovat. 
        \item Doplňkových tisknutelných materiálů, které jsou využitelné zejména v prezenční výuce. Každé z videí má svoji textovou verzi, která je obsahově stejná -- liší se pouze formou (místo audiovizuální písemná). Na konci každého návodu jsou umístěny doplňující otázky a úkoly, díky kterým si může student dané téma samostatně procvičit.
    \end{itemize}
    \item Vytvoření platformy (webového portálu), na kterém budou tyto materiály volně k~dispozici pro studenty i učitele
    \item Spolupráce s~vyučujícími strojírenských předmětů na implementaci vytvořených materiálů do výuky
\end{enumerate}
