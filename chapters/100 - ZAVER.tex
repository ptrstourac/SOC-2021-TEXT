\chapter*{Závěr}
\addcontentsline{toc}{chapter}{Závěr}
V průběhu práce jsem vytvořil komplexní výukovou sadu, kterou je možné snadno používat přímo ve výuce strojírenské konstrukce v~SolidWorks na středních průmyslových školách.
Celá tato výuková sada se skládá jednak z výukových videí sloužících jako audiovizuální instruktáž práce s různými prvky a funkcemi v programu SolidWorks.
Druhou, neméně důležitou částí jsou tisknutelné materiály, využitelné zejména při prezenční výuce, které obsahují doplňující otázky a úkoly.
Díky nim si může student dané téma zopakovat a vyučujícímu umožňuje ověřit znalosti tohoto studenta. 

Ke dni odevzdání této práce vzniklo již 11 výukových videí a pro každé z~nich existuje i textová verze obsažená v~tisknutelných doplňkových materiálech.
Tyto materiály jsou doplněny o~otázky a úkoly, sloužící k~procvičení daného tématu, případně ověření znalostí studenta.
Ve variantě těchto textových materiálů pro vyučující jsou navíc přidány doporučení k~aplikaci ve výuce, upozornění na problematické části postupu a řešení otázek a úkolů.

Již nyní jsou mé materiály volně dostupné na webovém portálu P3D (\href{https://www.p3dportal.cz}{www.p3dportal.cz}) a jsou při nynější distanční výuce využívána nejen studenty a vyučujícími na naší škole, ale i některými vysokoškoláky.
Aplikace materiálů do prezenční výuky je na naší škole plánována s~návratem studentů do škol.

Témat, která by si zasloužila začlenění do mé výukové sady, existuje obrovské množství.
Vzhledem k~tomu, že mne tato tvorba opravdu baví a zároveň má užitek pro studenty i vyučující, plánuji v~ní pokračovat i nadále.
Během následujících měsíců plánuji sadu rozšířit o~materiály zabývající se například tvorbou plechových dílů, svařovaných konstrukcí, prvků technické dokumentace a mnoha dalších.

Věřím, že mnou vytvořená výuková sada se bude dále rozrůstat a bude skvělým pomocníkem studentů i vyučujících.
\newpage