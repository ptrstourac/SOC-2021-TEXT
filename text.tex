\documentclass{template/socthesis}

\drafttrue
\jcmmfalse

\usepackage[T1]{fontenc} % evropské uvozovky
\usepackage{subcaption}
\usepackage{amsmath}
\usepackage{enumitem}
\usepackage{hyperref}
\usepackage{gensymb} % balíček symbolů
\usepackage{booktabs}
\usepackage{lmodern}
\usepackage{csquotes} % text lze uvést do uvozovek pomocí \enquote{text}, nepoužívat \enquote* !!!
\usepackage{textcomp}
\usepackage{pdfpages}

\usepackage[toc,page]{appendix}
\usepackage{color} % balíček pro obarvování textů
\usepackage{xcolor}  % zapne možnost používání barev, mj. pro \definecolor
\definecolor{mygreen}{RGB}{0,153,153} % nastavení barev odkazů 
\usepackage{listings} % balíček pro formátování zdrojových kódů 
\usepackage[author=,status=draft]{fixme} % vkládání poznámek  
% dva módy (status): draft (poznámky se zobrazují v PDF) / final (poznámky se nezobrazují v PDF)
\usepackage{graphicx}
\usepackage{multirow}
\usepackage{float}

\usepackage{expl3} % bibtex dependency, must be loaded prior to the bibtex
\usepackage[backend=bibtex,bibstyle=numeric,sorting=none,date=long,dateabbrev=false,texencoding=utf8,bibencoding=utf8,style=iso-numeric]{biblatex}

\usepackage[a4paper]{geometry}

\lstset { %
    language=C++,
    backgroundcolor=\color{black!5}, % set backgroundcolor
    basicstyle=\footnotesize,% basic font setting
}

\addbibresource{text.bib}
\nocite{*}

\titlecz{Sada materiálů pro podporu výuky strojírenské konstrukce v~SolidWorks}
\titleen{TITLEEN}
\author{Petr Štourač}
\field{12}
\school{Střední průmyslová škola a~Vyšší odborná škola Brno, Sokolská, příspěvková organizace}
\mentor{Ing. Václav Zavadil}
\mentorstatement{Ing. Václava Zavadila}

% Změňte, pokud se liší
%\region{Jihomoravský}
\placefooter{Brno 2021}

%\usepackage{hyperref} % balíček pro hypertextové odkazy
% \url{www.odkaz.cz}
% \href{http://www.odkaz.cz}{Text který bude jako odkaz}
% \hyperlink{label}{proklikávací_text} - odkaz na text 
% \hypertarget{label}{cíl_odkazu} - cíl odkazu 

\setlength{\parskip}{0.8em}
\begin{document}

\newgeometry{margin=2cm, top=4cm, bottom=2.5cm, left=2.5cm, includefoot}

\maketitle

\newgeometry{margin=2cm, top=2.5cm, bottom=2.5cm, left=2.5cm, right=2cm, includefoot}

\makecopyrightstatement{V~Brně}

\makethanks{\B{\textcolor{red}{PŠ: Poděkování sepísnu na závěr.}}}
% OLD: Děkuji svému školiteli Mgr. Miroslavu Burdovi za obětavou pomoc, podnětné připomínky a~hlavně nekonečnou trpělivost, kterou mi během práce poskytoval. Dále děkuji Kateřině Jelínkové za kontrolu gramatické správnosti a~korekce anglických textů. Kromě toho děkuji vyučujícím na naší škole za jejich podporu. Poděkování patří i Robotárně -- pobočce DDM Helceletova Brno za možnost využívání jejích prostor a~vybavení k~práci na SOČ.

\pagestyle{empty}

\section*{Anotace}
Počítačově asistovaný návrh je dnes nedílnou součástí strojírenské praxe.
Není proto divu, že se práce s~CAD programy běžně vyučuje na odborných školách s~technickým zaměřením.
Časová dotace těchto předmětů se zpravidla pohybuje okolo 2 až 4 hodin týdně, přičemž se liší jak mezi jednotlivými školami, tak i mezi obory.
Přesto, že se jedná o~jeden ze stěžejních předmětů, existuje pro něj velmi málo výukových materiálů.
Příprava výuky je tak čistě na samotných vyučujících.

Cílem této práce je usnadnit výuku konstrukce v~programu SolidWorks vytvořením edukativní sady zahrnující výukové videonávody, textové příručky a doplňkové materiály s~metodickými pokyny pro vyučující.

\subsection*{Klíčová slova}
SolidWorks, výuková sada, strojírenská konstrukce, výuková videa, podpora výuky

\vspace{20mm}

\section*{Annotation}
\B{\textcolor{red}{PŠ: Anotaci ještě přeložím, nicméně je to asi ta poslední věc na které teď záleží :-)}}

\subsection*{Keywords}
SolidWorks, educational kit, mechanical engineering, educational videos, teaching support

\newpage
\pagestyle{plain}

\setlength{\parskip}{0em}
\tableofcontents % vysází obsah

%%% Začátek práce
\setlength{\parskip}{0.8em}
\setcounter{figure}{0}
\setcounter{table}{0}
\newpage

% Uvod prace
\chapter*{Úvod}
\addcontentsline{toc}{chapter}{Úvod}
\begin{itemize}
    \item co jsou to CADy a proč se učí
    \item proč dělám to, co dělám
    \item co mají studenti aktuálně k dispozici
\end{itemize}

Počítačově asistovaný návrh je dnes nedílnou součástí strojírenské praxe.
Není proto divu, že se práce s CAD\footnote{Computer assisted design - počítačově asistovaný design} programy běžně vyučuje na odborných školách s technickým zaměřením.
Časová dotace těchto předmětů se zpravidla pohybuje okolo 2 až 4 hodin týdně, přičemž se liší jak mezi jednotlivými školami, tak i mezi obory.
Přesto, že se jedná o jeden ze stěžejních předmětů, existuje pro něj velmi málo výukových materiálů.
Příprava výuky je tak čistě na samotných vyučujících.

Pro výuku SolidWorks, který je jedním z nejčastěji vyučovaných CADů aktuálně existuje pouze jedna učebnice.
Videonávodů existuje sice mnohem více, zpravidla ale nejsou vhodné pro výuku na školách.



\newpage
 %%% Asi HOTOVO, možná ještě něco doplním...

\chapter{Cíle práce}

\begin{enumerate}
    \item 
\end{enumerate}
 %%% HOTOVO

\chapter{Dosavadní výuka strojírenské konstrukce}
Na začátek bych rád popsal, co to vlastně SolidWorks je, jak aktuálně probíhá výuka předmětů zaměřených na strojírenskou konstrukci v něm a materiály, které mají studenti k dispozici.

\section{Co to je SolidWorks a proč se používá?}
    CAD systém SolidWorks vyvíjený francouzskou společností Dassault Systémes je dnes jedním z nejpoužívanějších programů pro 3D modelování a tvorbu technické dokumentace ve strojírenství.
    Mimo již zmíněné funkce je možné s jeho pomocí vytvářet i různé simulace (pevnostní i pohybové), spravovat data jednotlivých výrobků a jejich životní cyklus (PLM\footnote{Product lifecycle management -- správa životního cyklu výrobku} a PDM\footnote{Product Data Management -- správa dat výrobku}), připravovat výrobní data pro obrábění (CAM\footnote{Computer Aided Manufacturing -- počítačová podpora výroby (resp. obrábění)}) a mnoho dalšího.
    Vzhledem k jeho komplexnosti jsou pro praktickou práci s ním zapotřebí alespoň základní znalosti funkcí tohoto systému.

    Právě kvůli rozšířenosti SolidWorks ve strojírenství se práce s ním běžně vyučuje na mnoha středních průmyslových školách.
    Podstatnou výhodu pro studenty tvoří možnost získání tzv. SDK\footnote{studentská licence bez jakýchkoliv doplň. modulů} (Student design kit), který jim umožňuje pracovat s 3D modely i doma.

    Za zmínku také stojí, že na některých školách mohou zdatnější studenti již v průběhu studia složit zkoušky pro získání certifikátu CSWA (Certified SolidWorks Associate), nebo CSWP (Certified SolidWorks Professional), díky kterým mohou později získat lepší pozici při žádání o práci ve firmě, která SolidWorks používá.

\section{Způsob výuky práce v SolidWorks na naší škole}
    Výuka již zmíněného konstrukčního cvičení probíhá na projektové bázi.
    Co si pod tímto pojmem představit?
    Vyučující seznámí studenty s projektem, na kterém budou v následujících týdnech popř. měsících pracovat. 
    Tyto projekty jsou v průběhu školního roku zpravidla 3 až 4.
    Následně jsou studenti seznámeni v průběhu několika hodin s postupem práce s prvky, které by mohli při práci na projektu potřebovat.
    Poté pracují samostatně na svých projektech.
    Ve vyučovacích hodinách proto tvoří samostatná práce studentů většinu času.

    Jestliže studenti pracují v průběhu vyučovací hodiny samostatně, mohou interagovat s vyučujícím a s jeho pomocí dospět ke správnému řešení.
    Mnozí studenti však na projektech pracují převážně doma, kde nemají možnost se jednoduše zeptat, pokud něčemu nerozumí, nebo nejsou schopni danou úlohu vyřešit.

\section{Výukové materiály související se strojírenskou konstrukcí}
    Pokud z výběru vyřadíme cizojazyčné publikace, existuje pouze jedna \It{Učebnice SolidWorks}\superscript{\cite{PAGAC}}, kterou tento seznam začíná a zároveň končí. 
    Videonávody na toto téma existují, zpravidla však neodpovídají školní praxi a často ani nejsou pro výuku vhodná.
    Většinou jsou dlouhá (> 10 minut), a student nemusí udržet záměrnou pozornost po celou dobu.
    Dalším problematickým aspektem je jejich obsáhlost.
    Pokud student hledá postup tvorby konkrétního prvku, není nutné, aby kvůli tomu zhlédnul celé dvacetiminutové video, ve kterém tvoří hledaný obsah například jen 5 procent. 
 %%% Kratší kapitola, dopíšu do 25. bř

\chapter{Názorně -- demonstrační pomůcky}
V této kapitole se zaměřuji na trendy ve výuce technických předmětů a metody při ní využívané.

\section{Trendy ve výuce technických předmětů}
Výuka technických předmětů vyžaduje velmi specifický přístup.
Aby student látku správně pochopil je kromě teoretických znalostí nezbytná i ukázka jejich praktické aplikace.
Nejběžnějším způsobem této ukázky jsou modely, díly, nebo mechanismy speciálně upravené tak, aby si student udělal komplexní představu o jejich funkci a účelu.
Tyto ukázkové modely však mají své nevýhody. 
Patrně největší z nich bývá patrná převážně při předvádění větších mechanismů.
Často totiž ve výuce není prostor a čas na detailní rozebrání modelu a studenti tak často vidí jen zevnějšek, nebo některé vnitřní části skrz speciálně prořezané otvory.

Stále častěji se proto využívá počítačových vizualizací, nebo animací, které umožňují celý model pomocí několika kliknutí snadno rozebrat, nebo některé části skrýt. 
Díky tomu mohou studenti vidět funkci některých mechanismů zevnitř, což by u modelu nemuselo být možné.

Častá bývá také praktická výuka vyučovaná v dílnách, popřípadě laboratořích, kde dochází k propojení teorie s praxí.
Pro příklad můžeme vzít třeba technologii obrábění.
Studenti se nejdříve v hodinách strojírenské technologie dozví, jak třískové obrábění probíhá, jeho princip a metody.
Následně si studenti mohou v dílnách toto obrábění vyzkoušet -- ať již na soustruhu, či na frézce. 

\section{Předvádění a pozorování}


\section{Instruktáž}
Ve školním prostředí je instruktáž jednou z~často používaných názorně -- demonstračních výukových metod.
Pomocí vizuálních, zvukových, popřípadě audiovizuálních podnětů umožňuje studentům si osvojovat nové dovednosti a v~kombinaci s~metodami praktickými je uplatňovat do praxe.
Skládá se zpravidla z~ukázky doplněné komentářem -- takzvanými instrukcemi.

\noindent V~závislosti na používaných podnětech lze rozlišit několik typů instruktáže:
\begin{itemize}[topsep=0pt]
    \setlength\itemsep{0em}
    \item \B{Slovní instruktáž} využívá verbálních instrukcí k~popsání vysvětlované činnosti.
    \item \B{Audiovizuální instruktáž} kombinuje slovní instruktáž s~praktickou ukázkou, nebo vizuálními podklady (obrázky, video).
    \item \B{Písemná instruktáž} je spojením slovní instruktáže a psané formy doplněné o~ilustrace.
\end{itemize}

\subsection{Instruktáž v~jednotlivých částech výukové sady}
Jak výuková videa, tak i vytisknutelné materiály tvoří určitou formu instruktáže.
V~případě výukových videí se jedná o~instruktáž audiovizuální, spojující mluvený komentář (slovní instrukce) s~názornou ukázkou.
Tyto dvě části se navzájem doplňují.
Zatímco názorná ukázka daný postup předvádí, mluvený komentář přidává doplňující informace a usměrňuje pozornost studenta.

Tisknutelné materiály jsou poté formou písemné instruktáže.
V~textové podobě jsou zde přítomny slovní pokyny doplněné o~ilustrace částí postupu, které jsou slovně špatně popsatelné (například vzhled určitého dialogového okna).

\section{Práce s~obrazem}

\newpage %%% Dopíšu do 2. dubna - bude potřeba intenzivní kopání

\chapter{Dílčí části výukové sady}
Výuková sada je složená ze čtyř navzájem na sebe navazujících částí.
\fxnote[inline=true]{\textcolor{mygreen}{Sem ještě něco dopíšu, jen zatím nevím co...}}

\section{Výuková videa}
\fxnote[inline=true]{\textcolor{mygreen}{Sem ještě něco dopíšu, jen zatím nevím co...}}

\subsection{Formát a struktura výukových videí}
Ještě před tím, než jsem začal vytvářet jednotlivá videa, jsem si musel odpovědět na několik důležitých otázek:
\begin{itemize}[topsep=0pt]
    \setlength\itemsep{0em}
    \item Jak budou videa koncipována? Bude se jednat o krátká videa zaměřená na jeden konkrétní prvek, nebo budou delší a zaměřená na širší problematiku?
    \item Kam budu hotová videa umisťovat?
    \item Jak budou videa vypadat po grafické i technické stránce?
    \item Jaké bude jejich využití a účel?
\end{itemize}

Ve snaze najít odpověď na první z nich jsem se zamyslel, jakým způsobem já sám vyhledávám informace.
Pokud potřebuji získat odpověď na konkrétní otázku v dlouhém textu, mám možnost využít textového vyhledávání. 
U videa ale žádná klávesová zkratka \It{Ctrl+F} zatím neexistuje -- musel bych tedy pomalu přeskakovat, až bych našel onu hledanou část.
Odpověď byla proto jasná -- krátká videa zaměřená na konkrétní prvek, jelikož díky nim budou studenti schopni najít řešení daného problému rychle a efektivně.

Úvaha nad druhou otázkou byla náročnější.
Na začátku jsem uvažoval nad umisťováním videí přímo na vlastní server, odkud by bylo možné je streamovat\footnote{Způsob přenosu dat, která jsou přenášena stabilním datovým proudem}.
V takovém případě bych nebyl vázaný limitací žádné služby a pokud by mi nějaká funkcionalita chyběla, mohl bych si ji snadno vyrobit.
Následně jsem si však uvědomil, že tato varianta by konečnému divákovi nepřinesla žádný užitek a proto jsem se rozhodl využít službu YouTube.
Její výhody pro koncového uživatele jsou stěžejní -- jedná se o velkou platformu, která je mezi studenty již velmi dobře známá.
Pro studenty by tedy nebyl žádný problém s orientací, nebo dostupností obsahu (YouTube využívá vlastní CDN\footnote{Content delivery network, globální síť serverů určených pro distribuci obsahu}, díky které je zajištěná téměř stoprocentní dostupnost).

Posledním, avšak nesporně zásadním bodem bylo zvážení grafického designu a technických parametrů.
Design sám o sobě prošel časem jistou proměnou, nicméně jsem se již od začátku snažil o to, aby videa vypadala moderně a čistě, což by studentům výrazně napomáhalo v orientaci v nich.
Po technické stránce jsem se rozhodl držet rozlišení 1920x1080 při 60 snímcích za vteřinu a hlasitosti zvukové stopy normalizované na -14 LUFS, což je standardní hlasitost pro videa nahrávaná na YouTube.
Těchto parametrů se od vydání prvního videa držím, aby byla všechna co možná nejjednotnější.

Mimo to bylo nutné připravit strukturu videí.
Vzhledem k tomu, že mají velmi stručně a srozumitelně představit optimální řešení daného problému, rozhodl jsem se v otázce struktury držet třech zásadních bodů:
\begin{itemize}[topsep=0pt]
    \setlength\itemsep{0em}
    \item Úvod videa, kde je problematika každého z nich stručně vysvětlena,
    \item hodnoty a parametry potřebné pro splnění úlohy,
    \item samotný postup tvorby daného prvku.
\end{itemize}
V úvodu je v krátkosti popsáno zaměření videa a smysl daného prvku, popř. postupu.
U videí zaměřených na modelování následuje výčet potřebných hodnot a parametrů, které jsou pro vytvoření prvku, nebo součásti nezbytné.
Nakonec následuje názorná komentovaná ukázka samotného postupu.

\subsection{Podkresová hudba}
Podstatné zlepšení dojmu z videí nastalo ve chvíli, kdy jsem začal do podkresu videa přidávat hudbu.
V samotných začátcích jsem čerpal skladby z platformy ncs.io, která poskytuje skladby k volnému užití za předpokladu uvedení autora a zdroje.
Postupně jsem však dospěl k závěru, že skladby použité u některých původních videí nejsou vzhledem k výukové povaze obsahu příliš vhodné a rozhodl se styl volené hudby změnit. 
Změnil jsem i zdroj hudby a začal čerpat z placené služby Artlist.io umožňující licencování obrovského množství hudby mnoha různých žánrů.

Při volbě skladby do podkresu se snažím volit žánry, které nebudou působit rušivě, příliš vážně, nebo naopak infantilně.
Hlasitost je vždy regulována tak, aby byl komentář ve videu dobře srozumitelný a hudba do něj příliš nezasahovala.

\subsection{Náhledové obrázky}
V případě videí P3D mají náhledové obrázky smysl hlavně z hlediska orientace.
Snažil jem se proto, aby měly všechny jednotné rozložení a design a lišily se pouze obsahem a barvami, nikoliv strukturou.
\begin{figure}[htbp]
    \centering
    \begin{minipage}[b]{0.45\textwidth}
        \centering
        \includegraphics[width=0.8\textwidth]{img/020/aktivace-realview-thumbnail.png}
        \caption{Instalace a nastavení}
        \label{fig:thumb1}
    \end{minipage}
    \qquad
    \begin{minipage}[b]{0.45\textwidth}
        \centering
        \includegraphics[width=0.8\textwidth]{img/020/perodr-hr-thumbnail.png}
        \caption{Modelování}
        \label{fig:thumb2}
    \end{minipage}
\end{figure}

\begin{figure}[htbp]
    \centering
    \begin{minipage}[b]{0.45\textwidth}
        \centering
        \includegraphics[width=0.8\textwidth]{img/020/pack-and-go-thumbnail.png}
        \caption{Sestavy}
        \label{fig:thumb3}
    \end{minipage}
    \qquad
    \begin{minipage}[b]{0.45\textwidth}
        \centering
        \includegraphics[width=0.8\textwidth]{img/020/dwg-perodr-hr-thumbnail.png}
        \caption{Výkresová dokumentace}
        \label{fig:thumb4}
    \end{minipage}
\end{figure}

Jak můžete vidět na jednotlivých náhledech \ref{fig:thumb1}, \ref{fig:thumb2}, \ref{fig:thumb3} a \ref{fig:thumb4}, každý z nich má v levém horním rohu umístěný nadpis s názvem série (resp. zaměřením), pod kterým se nachází konkrétní téma daného videa.
Na pravé polovině náhledu je v pozadí doplněný o obrázek ilustrující dané téma (například hřídel s drážkou pro pero).

Díky tomuto systému je studentovi již ve chvíli, kdy vidí náhled videa jasné jeho téma, což podstatně usnadňuje orientaci.
Zaměření videí jsou zároveň barevně odlišena, což umožňuje ještě rychlejší navigaci.
Při volbě barev jsem se snažil, aby nebylo možné je snadno zaměnit a byly vůči sobě dostatečně kontrastní.
Zvolené barvy mají za cíl od sebe témata barevně odlišit a utvořit tak na první pohled zjevné tematické celky.

\section{Tištěné materiály s otázkami a úkoly}
Samotná videa dokáží samostatně fungovat jako vzdělávací materiál, nicméně ne všem studentům může tato audiovizuální forma vyhovovat.
Proto jsem pro každé z videí vytvořil i psanou verzi vhodnou pro použití v prezenční výuce zejména v případě, kdy není možnost třídě video promítnout, nebo je žádoucí, aby studenti pracovali samostatně. 
Důležité je zmínit, že samotné tištěné návody jsou plnohodnotné a student na základě nich může úlohu splnit i bez zhlédnutí videa. 
\begin{figure}[htbp]
    \centering
    \begin{minipage}[b]{0.45\textwidth}
        \centering
        \includegraphics[width=0.75\textwidth]{img/020/guide1.png}
        \caption{Tištěné materiály}
        \label{fig:thumb3}
    \end{minipage}
    \qquad
    \begin{minipage}[b]{0.45\textwidth}
        \centering
        \includegraphics[width=0.75\textwidth]{img/020/guide2.png}
        \caption{Tištěné materiály}
        \label{fig:thumb4}
    \end{minipage}
\end{figure}


\subsection{Otázky a úkoly}
Na konci každého návodu na modelování jsou umístěny doplňující otázky a úkoly, které studentům umožňují si procvičit postupy, nebo ověřit získané znalosti.
U většiny videí z ostatních kategorií není zapotřebí znalosti ověřovat, jedná se o postupy, které studenti mohou používat, nicméně nejsou pro úspěšnou práci v SolidWorks nutné.
Řešení těchto otázek a úkolů jsou záměrně umístěny v sekci pro vyučující, která tvoří přílohu těchto tištěných materiálů.

\subsection*{Metodické pokyny}
Součástí přílohy tisknutelných materiálů pro vyučující jsou i doporučení pro vyučující pro využívání materiálů v prezenční i distanční výuce.

\subsection{Dostupnost tisknutelných materiálů}
Verze bez metodických pokynů a řešení úloh jsou (\fxnote[inline=true]{\textcolor{red}{BUDOU}}) volně k dispozici na webu \href{https://www.p3dportal.cz}{www.p3dportal.cz} v sekci \enquote{Ke stažení}.
O variantu pro vyučující je možné si zažádat na e-mailové adrese \href{mailto:info@parallaxproduction.cz}{info@parallaxproduction.cz}.

\section{Výukový portál P3D}
S narůstajícím počtem videí a dalších materiálů začal vznikat problém v přehlednosti a uváděním souvislostí.
Po delším přemýšlení jsem dospěl k závěru, že nejjednodušší řešení bude vytvořit pro projekt vlastní webové stránky.
Díky nim si mohou nejen studenti, ale i vyučující najít všechna videa a doplňkové materiály snadno a přehledně na jednom místě.

\subsection{Struktura webu}
Pro přehlednost jsem stránky na webu rozdělil do tří úrovní. 
V rámci celé struktury je pro snadnou navigaci zobrazena lišta s odkazy pro snadný přesun mezi stránkami.

\noindent\B{Úvodní stránka} funguje jako rozcestník k jednotlivým podstránkám. Je rozdělena na několik sekcí. V horní části se nachází úvodní grafika. Níže nalezneme dvě nejnovější videa vytvořená v rámci celého projektu P3D. Ještě níže jsou poté viditelné všechny kategorie videí. (Viz obrázky \ref{fig:p3dportal-hp1}, \ref{fig:p3dportal-hp2} a \ref{fig:p3dportal-hp3})

\noindent\B{Kategorie} jsou na webu aktuálně čtyři - 3D modelování, sestavy, výkresová dokumentace a instalace a nastavení. Na stránce každé kategorie jsou zobrazeny náhledové obrázky všech videí, které do dané kategorie patří. (Viz. \autoref{fig:p3dportal-cat})
Při kliknutí na některý z náhledových obrázků se otevře detail daného videa. 

\noindent\B{V detailu videa} je zobrazený jeho popis, potřebné hodnoty a parametry (pokud nějaké jsou) a doplňkové úkoly a otázky vč. jejich řešení. Úkoly s otázkami zobrazenými na detailu videa jsou jiné, než úkoly v tisknutelných materiálech. Nehrozí tak, že by si student odpověď našel na webu. (Viz )

\noindent

\begin{figure}[htbp]
    \centering
    \begin{minipage}[b]{0.45\textwidth}
        \centering
        \includegraphics[width=1\textwidth]{img/020/web/web-hp1.png}
        \caption{Úvodní grafika webu}
        \label{fig:p3dportal-hp1}
    \end{minipage}
    \qquad
    \begin{minipage}[b]{0.45\textwidth}
        \centering
        \includegraphics[width=1\textwidth]{img/020/web/web-hp2.png}
        \caption{Sekce \enquote{Nejnovější videa}}
        \label{fig:p3dportal-hp2}
    \end{minipage}
\end{figure}


\begin{figure}[htbp]
    \centering
    \begin{minipage}[b]{0.45\textwidth}
        \centering
        \includegraphics[width=1\textwidth]{img/020/web/web-hp3.png}
        \caption{Úvodní grafika webu}
        \label{fig:p3dportal-hp3}
    \end{minipage}
    \qquad
    \begin{minipage}[b]{0.45\textwidth}
        \centering
        \includegraphics[width=1\textwidth]{img/020/web/web-cat.png}
        \caption{Zobrazení celé kategorie}
        \label{fig:p3dportal-cat}
    \end{minipage}
\end{figure}

\begin{figure}[htbp]
    \centering
    \begin{minipage}[b]{0.45\textwidth}
        \centering
        \includegraphics[width=1\textwidth]{img/020/web/web-3D.png}
        \caption{Detail videa}
        \label{fig:p3dportal-3D}
    \end{minipage}
    \qquad
    \begin{minipage}[b]{0.45\textwidth}
        \centering
        \includegraphics[width=1\textwidth]{img/020/web/web-assembly.png}
        \caption{Detail videa}
        \label{fig:p3dportal-assembly}
    \end{minipage}
\end{figure}

\subsection{Design webu}
Stejně jako u videí jsem se zaměřil na to, aby byl design webového portálu moderní a čistý.
Pozadí a většina prvků webu je laděné do tmavých barev.
Toto rozhodnutí jsem učinil z několika důvodů.
Jednak je sledování menšího počtu světlejších elementů na tmavém pozadí příjemnější pro oči (zvláště v pozdních hodinách), jednak protože jsou v dnešní době tmavé vzhledy u aplikací a webových stránek velice populární.
Dále jsem vycházel z mé vlastní preference tmavých barev.

Volba ostatních barev vychází z designu náhledových obrázků videí, opět pro zachování konzistence.

\subsection{Zkracovací subdoména go.p3dportal.cz}
Přestože jsou všechny odkazy na webu snadno čitelné\footnote{Sestávají se z dobře čitelných částí slov, nikoliv náhodných řetězců znaků}, kvůli struktuře webu mohou být příliš dlouhé.
Z toho důvodu jsem vytvořil tzv. zkracovací subdoménu \href{https://go.p3dportal.cz}{go.p3dportal.cz}, která umožňuje jakémukoliv odkazu přiřadit jeho zkrácený alias.
Běh této funkce je zajišťován white-labelovým\footnote{Produkt, který si jeho uživatel může skrýt pod vlastní frontend, v tomto případě doménu} zkracovačem \href{https://short.io}{short.io}.
Doména go.p3dportal.cz tak slouží čistě jako front end\footnote{Front end = vnější vrstva určité aplikace (to, co vidí uživatel), opakem je back end}.
 %%% Napíšu do 24. bř, možná ještě něco doplním později

\chapter{Integrace do výuky a využití}
Již v~průběhu psaní této práce jsou materiály z~výukové sady využívány vyučujícími a studenty naší školy.
Kromě nich je využívají i někteří vysokoškoláci a nestudující lidé, kteří se chtějí naučit modelovat.

\section{Sledovanost v~číslech}
\B{\textcolor{mygreen}{PŠ: Tuto sekci bych rád vztáhnul ke konkrétnímu datu a zmínil konkrétní čísla. Rád bych počkal ještě nějakou chvíli, mezitím by se sledovanost měla přehoupnout přes ty dva tisíce. TZN napíšu nejpozději ve středu večer... }\normalsize}

\section{Distanční výuka}
Výuková videa našla při distanční výuce podstatné využití.
Videokonferenční programy nezaručují dostatečně vysokou kvalitu přenosu obrazu ani zvuku, kvůli čemuž by musel vyučující ukázku postupu při výpadku na straně studentů často opakovat.
Učitelé na naší škole proto v~některých případech vlastní ukázky zcela nahrazují odkazováním na mnou vytvořená výuková videa a materiály, jelikož jsou schopny tyto ukázky plně zastoupit.
Sami studenti tyto materiály využívají při práci na svých projektech, jelikož kvůli absenci prezenční výuky nemají možnost pracovat v~hodinách za přítomnosti učitele a nemohou se na něj tedy obrátit s~žádostí o~pomoc.
Ti zvídavější z~nich se pomocí mých materiálů mohou učit nové postupy, ke kterým v~běžné výuce ještě nedošli.

\section{Využití materiálů v~prezenční výuce}
K~praktickému využití v~prezenční výuce bohužel z~důvodu celosvětové pandemie zatím nedošlo. 
I~přesto je jejich aplikace do výuky strojírenské konstrukce plánovaná přinejmenším na naší škole.
Většina vyučujících je začne s~obnovením prezenční výuky využívat a v~příštím roce bych praktickou aplikaci svých materiálů dále rozšířil.

Vzhledem k~tomu, že jsou všechny materiály volně dostupné, jejich využití je proto možné i na jiných školách.
S~tím se pojí i velmi snadné začlenění do výuky.
Vyučující nemusí nic složitě hledat -- na webových stránkách snadno a rychle najde vytisknutelné materiály i výuková videa.
Textové materiály samozřejmě není nutné používat jen v~papírové podobě. Studentům je může snadno rozeslat v~PDF, popřípadě může jen odkázat na webové stránky \href{https://www.p3dportal.cz}{www.p3dportal.cz}.

Způsobů využití mojí výukové sady v~prezenční výuce existuje celá řada.
Vyučující upřednostňující textovou formu mohou studentům vytisknout a rozdat materiály, nebo jim je snadno rozeslat.
Pokud je v~učebně k~dispozici projektor, mohou promítat výuková videa, případně nechat studenty, aby je sledovali samostatně na svých počítačích.
Další možností je kombinované využití výukových videí a tištěných materiálů, kdy vyučující pomocí projektoru nechá studenty zhlédnout video, následně jim rozdá vytisknuté materiály a nechá je vypracovat úkoly a otázky, které jsou v~nich obsaženy. %%% Dopíšu do 2. dubna

% Zaver prace
\chapter*{Závěr}
\addcontentsline{toc}{chapter}{Závěr}
Když jsem na tomto projektu začínal pracovat jsem netušil, že se takto rozroste.
Z~původního cíle vytvořit pár výukových videí vznikla komplexní výuková sada, kterou je možné snadno používat přímo ve výuce strojírenské konstrukce v~SolidWorks na středních průmyslových školách.

Ke dni odevzdání této práce vzniklo již 11 výukových videí a pro každé z~nich existuje i textová verze obsažená v~tisknutelných doplňkových materiálech.
Tyto materiály jsou doplněny o~otázky a úkoly, sloužící k~procvičení daného tématu, případně ověření že student látku zvládnul.
Ve variantě těchto textových materiálů pro vyučující jsou navíc přidány doporučení k~aplikaci ve výuce, upozornění na problematické části postupu a řešení otázek a úkolů.

Již nyní jsou mé materiály volně dostupné na webovém portálu P3D (\href{https://www.p3dportal.cz}{www.p3dportal.cz}) a jsou při nynější distanční výuce využívána nejen studenty a vyučujícími na naší škole, ale i některými vysokoškoláky.
Aplikace materiálů do prezenční výuky je na naší škole plánována s~návratem studentů do škol.

Témat, která by si zasloužila začlenění do mé výukové sady, existuje obrovské množství.
Vzhledem k~tomu, že mne tato tvorba opravdu baví a zároveň má užitek pro studenty i vyučující, plánuji v~ní pokračovat i nadále.
Během následujících měsíců plánuji sadu rozšířit o~materiály zabývající se například tvorbou plechových dílů, svařovaných konstrukcí, prvků technické dokumentace a mnoha dalších.

\newpage %%% Jako vždy píšu až na konec...
\newpage

\appendix
\addcontentsline{toc}{chapter}{Přílohy}

% Prilohy
\chapter{Seznam již vydaných videí} \label{released-videos}
Tato příloha obsahuje kompletní seznam videí vzniklých v~rámci projektu P3D vč. odkazů rozdělených dle jednotlivých témat. \newline
\noindent\It{Pozn.: při kliknutí na odkaz budete přesměrování na stránku korespondujícího videa}.

\section{Instalace a zprovoznění SolidWorks SDK} \label{videa-instalace}
\href{https://go.p3dportal.cz/inst-sdk2021}{Instalace a první spuštění SolidWorks SDK 2020/2021 (go.p3dportal.cz/inst-sdk2021)} \newline
\href{https://go.p3dportal.cz/sablony-vid}{Instalace šablon a knihoven norm. dílů ze Sokolské (go.p3dportal.cz/sablony-vid)} \newline
\href{https://go.p3dportal.cz/GkkCtx}{Aktivace Realview na necertifikované grafické kartě (go.p3dportal.cz/realview)} \newline

\section{Základy modelování} \label{videa-modelovani}
\href{https://go.p3dportal.cz/pruzina}{Jednoduchá pružina (go.p3dportal.cz/pruzina)} \newline
\href{https://go.p3dportal.cz/oz-kolo}{Ozubené kolo s~přímým čelním ozubením (go.p3dportal.cz/oz-kolo)} \newline
\href{https://go.p3dportal.cz/vykresove-ozk}{Ozubené kolo pro výkres - obálka (go.p3dportal.cz/vykresove-ozk)} \newline
\href{https://go.p3dportal.cz/jr-rk}{Jednořadé řetězové kolo (go.p3dportal.cz/jr-rk)} \newline
\href{https://go.p3dportal.cz/perodr-na}{Drážka pro pero v~náboji (go.p3dportal.cz/perodr-na)} \newline
\href{https://go.p3dportal.cz/perodr-hr}{Drážka pro pero na hřídeli (go.p3dportal.cz/perodr-hr)} \newline

\section{Práce se sestavami} \label{videa-sestavy}
\href{https://go.p3dportal.cz/prejm-dilu}{Přejmenování dílu v~sestavě (go.p3dportal.cz/prejm-dilu)} \newline
\href{https://go.p3dportal.cz/pack-and-go}{Přesun sestavy pomocí Pack and Go (go.p3dportal.cz/pack-and-go)} \newline

\chapter{Tisknutelné materiály}
Součástí příloh této práce jsou i doplňkové materiály ve variantě pro studenty i pro vyučující.
Vzhledem k~tomu, že s~přibývajícím množstvím výukových videí se rozrůstají i textové materiály, je jejich pravidelně aktualizovaná verze k~dispozici na \href{https://go.p3dportal.cz/textmat-st}{go.p3dportal.cz/textmat-st} (studenti) a \href{https://go.p3dportal.cz/textmat-uc}{go.p3dportal.cz/textmat-uc} (vyučující).
Verze přiložená k této práci je aktuální k~datu 8. 4. 2021.


\setlength{\parskip}{0em}
\printbibliography[title=Literatura]
\addcontentsline{toc}{chapter}{Literatura}

\listoffigures
\addcontentsline{toc}{section}{Seznam obrázků}

%\listoftables
%\addcontentsline{toc}{section}{Seznam tabulek}

\setlength{\hoffset}{0mm}
%%% ODKOMENTOVAT PŘED FINÁLNÍ KOMPILACÍ!!!
\addcontentsline{toc}{chapter}{Příloha: Tisknutelné materiály pro studenty}
\includepdf[pages=-]{P3D-auxmaterials/text.pdf}

Zde budou ještě materiály pro učitele...
\addcontentsline{toc}{chapter}{Příloha: Tisknutelné materiály pro vyučující}

\end{document}
