\documentclass{template/socthesis}

\drafttrue
\jcmmfalse

%\renewcommand{\familydefault}{\sfdefault} %% Only if the base font of the document is to be sans serif
%\usepackage{bera}

\usepackage[T1]{fontenc} % evropské uvozovky
\usepackage{subcaption}
\usepackage{amsmath}
\usepackage{enumitem}
\usepackage{hyperref}
\usepackage{gensymb} % balíček symbolů
\usepackage{booktabs}
\usepackage{lmodern}
\usepackage{csquotes} % text lze uvést do uvozovek pomocí \enquote{text}
\usepackage{textcomp}

\usepackage[toc,page]{appendix}
\usepackage{color} % balíček pro obarvování textů
\usepackage{xcolor}  % zapne možnost používání barev, mj. pro \definecolor
\definecolor{mygreen}{RGB}{0,153,153} % nastavení barev odkazů 
\usepackage{listings} % balíček pro formátování zdrojových kódů 
\usepackage[author=,status=final]{fixme} % vkládání poznámek  
% dva módy (status): draft (poznámky se zobrazují v PDF) / final (poznámky se nezobrazují v PDF)
\usepackage{graphicx}
\usepackage{multirow}
\usepackage{float}

\usepackage{expl3} % bibtex dependency, must be loaded prior to the bibtex
\usepackage[backend=bibtex,bibstyle=numeric,sorting=none,date=long,dateabbrev=false,texencoding=utf8,bibencoding=utf8,style=iso-numeric]{biblatex}

\usepackage[a4paper]{geometry}

\lstset { %
    language=C++,
    backgroundcolor=\color{black!5}, % set backgroundcolor
    basicstyle=\footnotesize,% basic font setting
}

\addbibresource{text.bib}
\nocite{*}

\titlecz{Videonávody pro výuku konstrukce v SolidWorks}
\titleen{Videoguides for SolidWorks construction education}
\author{Petr Štourač}
\field{12}
\school{Střední průmyslová škola a~Vyšší odborná škola Brno, Sokolská, příspěvková organizace}
\mentor{Mgr. Miroslav Burda}
\mentorstatement{Mgr. Miroslava Burdy}

% Změňte, pokud se liší
%\region{Jihomoravský}
\placefooter{Brno 2021}

%\usepackage{hyperref} % balíček pro hypertextové odkazy
% \url{www.odkaz.cz}
% \href{http://www.odkaz.cz}{Text který bude jako odkaz}
% \hyperlink{label}{proklikávací_text} - odkaz na text 
% \hypertarget{label}{cíl_odkazu} - cíl odkazu 


\begin{document}

\newgeometry{margin=2cm, top=4cm, bottom=2.5cm, left=2.5cm, includefoot}

\maketitle

\newgeometry{margin=2cm, top=2.5cm, bottom=2.5cm, left=2.5cm, right=2cm, includefoot}

\makecopyrightstatement{V~Brně}

\makethanks{}
% OLD: Děkuji svému školiteli Mgr. Miroslavu Burdovi za obětavou pomoc, podnětné připomínky a~hlavně nekonečnou trpělivost, kterou mi během práce poskytoval. Dále děkuji Kateřině Jelínkové za kontrolu gramatické správnosti a~korekce anglických textů. Kromě toho děkuji vyučujícím na naší škole za jejich podporu. Poděkování patří i Robotárně -- pobočce DDM Helceletova Brno za možnost využívání jejích prostor a~vybavení k~práci na SOČ.

\pagestyle{empty}

\section*{Anotace}


\subsection*{Klíčová slova}


\vspace{20mm}

\section*{Annotation}

%\fxnote[author=PŠ]{Přeložím během dnešního večera}

\subsection*{Keywords}


\newpage
\pagestyle{plain}

\tableofcontents % vysází obsah

%%% Začátek práce
\setcounter{figure}{0}
\setcounter{table}{0}
\newpage

% Uvod prace
\chapter*{Úvod}
\addcontentsline{toc}{chapter}{Úvod}
Úvod práce má za cíl uvést:
\begin{itemize}
    \item cíl práce
    \item jak ho chcete dosáhnout
    \item popis tématu práce, musí být výstižný, ale stručný a poutavý
\end{itemize}

Úvodu a závěru práce je třeba věnovat obzvláště velkou pozornost.
Myslete na to, že úvod a někdy i závěr si porotce čte jako první, teprve potom, jestli ho práce zaujme se rozhodne, zda ji přečte celou.
\newpage


% Motivace
\chapter{Motivace}

\newpage

% Zaver prace
\chapter*{Závěr}
V závěru by mělo být:
\begin{itemize}
    \item Rekapitulace cíle práce
    \item Dosáhnul jsem jej? Ano, nebo ne?
    \item Zhodnocení průběhu práce
    \item Co mi práce dala?
\end{itemize}

\newpage
\newpage

\appendix
\addcontentsline{toc}{chapter}{Přílohy}

% Prilohy
\chapter{Obrazové přílohy}


\newpage

\printbibliography[title=Literatura]
\addcontentsline{toc}{chapter}{Literatura}

\listoffigures
\addcontentsline{toc}{section}{Seznam obrázků}

\listoftables
\addcontentsline{toc}{section}{Seznam tabulek}

\end{document}
